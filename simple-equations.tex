\documentclass[]{article}

\usepackage[utf8]{inputenc}
\usepackage{amsmath}
\usepackage{amssymb}
\usepackage{hyperref}

%opening
\title{}
\author{}

\begin{document}

\maketitle

\begin{abstract}
Some examples of Mathematical Equations rendered in LaTex
\end{abstract}


\section{Algebra}

The well known Pythagorean theorem \(x^2 + y^2 = z^2\) was 
proved to be invalid for exponents greater than $2$. 

Thus the more general equation has no integer solutions:

\[ x^n + y^n = z^n \]


\begin{equation} \label{eq1}
	\begin{split}
		A & = \frac{\pi r^2}{2} \\
		& = \frac{1}{2} \pi r^2
	\end{split}
\end{equation}

\subsection{Aligning Equations}

\begin{align*}
	a+b \times a-b  &=  (a+b)(a-b) &= a^2 - b^2 \\ 
	x+y \times x-y  &=  (x+y)(x-y) &= x^2 - y^2 \\
	p+q \times p-q  &=  (p+q)(p-q) &= p^2 - q^2
\end{align*}


\begin{gather*} 
	(a+b)^2 = a^2 + 2ab + b^2 \\ 
	(a-b)^2 = a^2 - 2ab + b^2
\end{gather*}

\begin{align*}
	f(x) &= x^2 \\
	g(x) &= \frac{1}{x} \\
	F(x) &= \int^a_b \frac{1}{3}x^3
\end{align*}

\subsection{Various Examples}

\[ 
F = G \left( \frac{m_1 m_2}{r^2} \right)
\]

\[ 
\left[  \frac{ N } { \left( \frac{L}{p} \right)  - (m+n) }  \right]
\]


\begin{equation*}
	y = 1 + \left(  \frac{1}{x} + \frac{1}{x^2} + \frac{1}{x^3} +  \ldots  \\
             + \frac{1}{x^{n-1}} + \frac{1}{x^n} \right)
\end{equation*}


\[ 
\left \{
\begin{tabular}{ccc}
	1 & 5 & 8 \\
	0 & 2 & 4 \\
	3 & 3 & -8 
\end{tabular}
\right \}
\]

\newpage

\section{Trigonometry}

\[
\sin^2(\theta)+\cos^2(\theta) = 1
\]

\section{Operators}

	\[
\lim_{h \rightarrow 0 } \frac{f(x+h)-f(x)}{h}
\]
This operator changes when used alongside 
text \quad  \( \lim_{x \rightarrow h} (x-h) \).


\begin{eqnarray*}
	f(x) = \sum_{i=0}^{n} \frac{a_i}{1+x} \\
	\textstyle f(x) = \textstyle \sum_{i=0}^{n} \frac{a_i}{1+x} \\
	\scriptstyle f(x) = \scriptstyle \sum_{i=0}^{n} \frac{a_i}{1+x} \\
	\scriptscriptstyle f(x) = \scriptscriptstyle \sum_{i=0}^{n} \frac{a_i}{1+x}
\end{eqnarray*}


\subsection{Fractions and Binomials}

\[ \frac{1+\frac{x}{y}}{1+\frac{1}{1+\frac{1}{z}}} \]

\[ \frac{1}{\sqrt{x}} \]

\[
a_0+{1\over\displaystyle a_1+
	{1\over\displaystyle a_2+
		{1 \over\displaystyle a_3 + 
			{1 \over\displaystyle a_4}}}}
\]

Mixing text and equations:

\[ f(x)=\frac{P(x)}{Q(x)} \ \ \textrm{and} 
\ \ f(x)=\textstyle\frac{P(x)}{Q(x)} \]


\[
\binom{n}{k} = \frac{n!}{k!(n-k)!}
\]\\

\section{Matrices}

The identity matrix:

\[
\left[
\begin{matrix}
	1 & 0 & 0 \\
	0 & 1 & 0 \\
	0 & 0 & 1
\end{matrix}
\right]
\]

\subsection{Greek Letters}
\newcommand\omicron{o}
 
$$ \alpha \quad \beta \quad  \gamma \quad \delta \  \epsilon \   \zeta \  \eta \  \theta \  \iota \  \kappa \  \lambda \  \mu \  \nu  \  \xi \  
\omicron \ \pi \  \rho \  \sigma \  \tau \  \upsilon \   \phi \  \chi \  \psi \  \omega $$


\subsection{Binary Operators}

$$ \times \quad \otimes \quad \oplus \quad \cup \quad \cap $$

\subsection{Relation Operators}

$$ < \quad > \quad \subset \quad \supset \quad \subseteq \quad \supseteq $$

\subsection{Other Symbols}

$$ \int \quad \oint \quad \sum \quad \prod $$

\begin{itemize}
\item  $\mathbb{P}$ \quad Prime numbers
\item $ \mathbb{W} \quad \textrm{Whole numbers} $
\item $ \mathbb{N} \quad \textrm{Natural numbers} $
\item $ \mathbb{Z} \quad \textrm{Integers} $
\item $ \mathbb{Q} \quad \textrm{Rational numbers} $
\item $ \mathbb{R} \quad \textrm{Real numbers} $
\item $ \mathbb{C} \quad \textrm{Complex numbers} $
\item $ \mathbb{H} \quad \textrm{Quaternions} $
\item $ \mathbb{O} \quad \textrm{Octonions} $
\item $ \mathbb{S} \quad \textrm{Sedenions} $
\end{itemize}


%% \[
%% \N \quad
%% \natnums \quad
%% \Z \quad
%% \R \quad
%% \reals \quad
%%\Reals \quad
%% \cnums \quad
%% \Complex \quad
%% \]

\[
\langle \big\langle \bigg\langle \Bigg\langle \Bigg\rangle \bigg\rangle  \big\rangle \rangle
\]



\newpage


\section{Complex Numbers}

\[ i^2  -1 \]

\[ z = x + i y \]
  
$$ \mathbb{C} $$
$$ \{z,\overline{z}\} \in \mathbb{C} $$
$$ \Re(z)=a $$
$$ \Im(z)=b $$

$$ \Re(z),\Im(z) $$ 
$$ \operatorname{Re}(z), \operatorname{Im}(z) $$ 
$$ \mathfrak{Re}(z), \mathfrak{Im}(z) $$ 

$$ z=a+ib $$ 
$$ \bar{z}=a-ib $$ 
$$ \Re{z}=a $$
$$ \Im{z}=b $$

$$ \bar{z_{1}z_{2}} $$
$$ \overline{z_{1}z_{2}} $$

\newpage


\section{Navier-Stokes Equations}

\subsection{Einstein Summation Convention}

\begin{equation}
	\frac{\partial \rho}{\partial t} + \frac{\partial(\rho u_{i})}{\partial x_{i}} = 0
\end{equation}

\begin{equation}
	\frac{\partial (\rho u_{i})}{\partial t} + \frac{\partial[\rho u_{i}u_{j}]}{\partial x_{j}} = -\frac{\partial p}{\partial x_{i}} + \frac{\partial \tau_{ij}}{\partial x_{j}} + \rho f_{i} \end{equation}
\begin{equation}
	\frac{\partial (\rho e)}{\partial t} + (\rho e+p)\frac{\partial u_{i}}{\partial x_{i}} = \frac{\partial(\tau_{ij}u_{j})}{\partial x_{i}} + \rho f_{i}u_{i} + \frac{\partial(\dot{ q_{i}})}{\partial x_{i}} + r \end{equation}
The Einstein summation convention dictates that: When a sub-indice (here $i$ or $j$) is twice or more repeated in the same equation, one sums across the n-dimensions. 

So, in the context of Navier-Stokes in three spatial dimensions,  one repeats the term three times, each time changing the indice for one representing the corresponding dimension (ie $1,2,3$ or $x,y,z$).

The first equation is therefore a shorthand representation of: $\frac{\partial \rho}{\partial t}+\frac{\partial(\rho u_{1})}{\partial x_{1}}+\frac{\partial(\rho u_{2})}{\partial x_{2}}+ \frac{\partial(\rho u_{3})}{\partial x_{3}}=0$.

The second equation is actually a superposition of three separable equations which could be written in a three-line form: one line equation for each $i$ in each of which one sums the three terms for the $j$ sub-indice.

\subsection{Classic Notation with $\longrightarrow , \otimes , \nabla$}

\begin{equation}
	\frac{\partial \rho}{\partial t} + \overrightarrow{\nabla}\cdot(\rho\overrightarrow{u})=0
\end{equation}


\begin{equation}
	\frac{\partial(\rho \overrightarrow{u})}{\partial t} + \overrightarrow{\nabla}\cdot[\rho\overline{\overline{u\otimes u}}] = -\overrightarrow{\nabla p} + \overrightarrow{\nabla}\cdot\overline{\overline{\tau}} + \rho\overrightarrow{f}
\end{equation}

\begin{equation}
	\frac{\partial(\rho e)}{\partial t} + \overrightarrow{\nabla}\cdot((\rho e + p)\overrightarrow{u}) = \overrightarrow{\nabla}\cdot(\overline{\overline{\tau}}\cdot\overrightarrow{u}) + \rho\overrightarrow{f}\overrightarrow{u} + \overrightarrow{\nabla}\cdot(\overrightarrow{\dot{q}})+r
\end{equation}

Here $\otimes$ denotes the tensor product which forms a tensor from the constituent vectors.  A double bar denotes a tensor.  These equations using the Classic Notation are equivalent to the earlier three using the Einstein Summation Convention.

\subsection{Navier Stokes Equation 3D Cartesia Coordinates}

\begin{center}
	\bigskip\LARGE{$x: \rho \: (\partial_t u_x + u_x \: \partial_x u_x + u_y \partial_y u_x + u_z \: \partial_z u_x) = -\partial_x p \; + \mu \: (\partial_x^2 u_x \; + \partial_y^2 u_x \; + \partial_z^2 u_x) + \frac{1}{3} \mu \partial_x \: (\partial_x^2 u_x \; + \partial_y^2 u_y \; + \partial_z^2 u_z) + pg_x$}\\
	\bigskip\LARGE{$y: \rho \: (\partial_t u_y + u_x \: \partial_x u_y + u_y \partial_y u_y + u_z \: \partial_z u_y) = -\partial_y p \; + \mu \: (\partial_x^2 u_y \; + \partial_y^2 u_y \; + \partial_z^2 u_y) + \frac{1}{3} \mu \partial_y \: (\partial_x^2 u_x \; + \partial_y^2 u_y \; + \partial_z^2 u_z) + pg_y$}\\
	\bigskip\LARGE{$z: \rho \: (\partial_t u_z + u_x \: \partial_x u_z + u_y \partial_y u_z + u_z \: \partial_z u_z) = -\partial_z p \; + \mu \: (\partial_x^2 u_z \; + \partial_y^2 u_z \; + \partial_z^2 u_z) + \frac{1}{3} \mu \partial_z \: (\partial_x^2 u_x \; + \partial_y^2 u_y \; + \partial_z^2 u_z) + pg_z$\\}
	
\end{center}
\newpage

\section{Logarithhms}

{\fontsize{30pt}{36pt}\selectfont
	\[
	\LARGE{ \log x y = \log x + \log y
	}
	\]
	
\section{Calculus}

{\fontsize{30pt}{36pt}\selectfont
\[
\LARGE{
	\frac{df}{dt} = \lim_{h \rightarrow 0} \frac{f(t+h) - f(t)}{h}
}
\]
	
\section{Newton's Law of Gravity}

{\fontsize{30pt}{36pt}\selectfont
	\[
	\LARGE{
		F = G \frac{m_1 m_2}{d^2}
	}
	\]
	
}\section{Complex Numbers}

{\fontsize{30pt}{36pt}\selectfont
\[
\LARGE{
	i^2 = -1
}
\]
}

\section{Euler's Formula for Polyhedra}

{\fontsize{30pt}{36pt}\selectfont
	\[
	\LARGE{
		F - E + V = 2
	}
	\]
}

\section{Maxwell's Equations}

\section{The Normal Distribution}

{\fontsize{30pt}{36pt}\selectfont
	\[
	\LARGE{
		\Phi(x) =  \frac{1}{\sqrt{2 \pi \sigma}} e^{- \frac{(x - \mu)^2}{2 \sigma^2}}
		}
	\]
}

\section{The Wave Equation}

{\fontsize{30pt}{36pt}\selectfont
\[
\LARGE{\frac{\partial{{^2}u}}{{\partial{t}^2}} = c^2 \frac{\partial{{^2}u}}{{\partial{x}^2}}}
\]
}

\section{The Fourier Transform}

{\fontsize{30pt}{36pt}\selectfont
\[
\LARGE{\hat{f}(\xi) = \int^{\infty}_{-\infty} f(x) e^{2\pi i x \xi} \quad dx}
\]
 }


\newpage

\section{The Second Law of Thermodynamics}

{\fontsize{30pt}{36pt}\selectfont
\[
	\LARGE{
	dS \geq 0
	}
\]
}

\section{Relativity}

{\fontsize{30pt}{36pt}\selectfont
	\[
	E = {mc^2} 
	\]
}

This equation states Einstein's mass energy equivalence relationship.

\section{Relativity}

{\fontsize{30pt}{36pt}\selectfont
	\[
	i \hbar \frac{\partial{}}{\partial{t}}\Psi = \hat{H}\Psi
\]
}

\section{Information Thoery}

{\fontsize{30pt}{36pt}\selectfont
	\[
	H = - \sum_{x} p(x) log \ p(x) 
	\]
}

\section{Chaos Thoery}

{\fontsize{30pt}{36pt}\selectfont
	\[
	x_{t+1} = kx_t ( 1 -x_t )
	\]
}

\section{Black-Scholes Equation}

{\fontsize{30pt}{36pt}\selectfont
	\[
	\frac{1}{2} \sigma^2 S^2 \frac{\partial{{^2}V}}{\partial{S^2}}+ r S \frac{\partial{V}}{\partial{S}} + \frac{\partial{V}}{\partial{t}} - r V = 0
	\]
}

\newpage

\section{References}

\begin{itemize}
	\item \url{https://www.overleaf.com/learn/latex/Mathematical_expressions}
	\item\url{https://github.com/EdwardCalzia/Latex-Formulas}
	\item \url{https://latex-programming.fandom.com/wiki/List_of_LaTeX_symbols}
	\item \url{https://oeis.org/wiki/List_of_LaTeX_mathematical_symbols}
\end{itemize}


\end{document}
