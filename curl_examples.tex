\documentclass{article}

\usepackage{physics}
\usepackage{amsmath}

\begin{document}


In \LaTeX, the best practice is to use the physics package for curl symbol as well, because the physics package contains a pre-defined \\curl command  $\curl$  that denotes the entire curl operator.
	
	
	$$ \curl $$
	$$ \curl{\vb{F}} $$
	
	$$ \curl(\vb{F_{1}}+\vb{F_{2}}) $$
	$$ \qty(\pdv{x}\hat{\imath}+\pdv{y}\hat{\jmath}+\pdv{z}\hat{k})\cp \vb{F} $$

\[
\nabla \cp \vb{F}= \begin{vmatrix}
\vu{\imath} & \vu{\jmath} & \hat{k} \\ 
\pdv{x} & \pdv{y} & \pdv{z} \\ 
F_{x} & F_{y} & F_{z}
\end{vmatrix}
\]

\newcommand{\rx}{\mathbb{R}}

% \[ \rx  \]


\newcommand{\bb}[1]{\mathbb{#1}}


Other numerical systems have similar notations.
 
% The complex numbers \( \bb{C} \), the rational 
% numbers \( \bb{Q} \) and the integer numbers \( \bb{Z} \).

\section{Another Example}

\newcommand{\plusbinomial}[3][2]{(#2 + #3)^#1}

To save some time when writing too many expressions 
with exponents is by defining a new command to make simpler:

\[ \plusbinomial{x}{y} \]

And even the exponent can be changed

\[ \plusbinomial[4]{y}{y} \]

\end{document}